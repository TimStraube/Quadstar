\section{Einleitung} 
Quadcopter sind vielseitige Flugsysteme, welche die Luftfahrt revolutioniert haben und in einer stetig wachsenden Zahl an Anwendungen zum Tragen kommen. Diverse Forschungsprojekte weltweit beschäftigen sich mit der Entwicklung, Implementierung und Optimierung von Teil- und Gesamtsystemen. So beispielsweise eine Gruppe in Kanada, welche ein Programm zur Flugdynamiksimulation eines generischen Quadcopters veröffentlicht hat \ref{link:SimCon}. Am Institut für Dynamische Systeme und Regelung der ETH Zürich wurde ein Quadcopterregler entwickelt und in einer Veröffentlichung beschrieben \ref{link:ETH}. Dieser baut auf einer kaskadierten Architektur auf, funktioniert mit Einheitsquaternionen und verspricht hohe Performance besonders in dynamischen Lagen auch elf Jahre nach der Veröffentlichung. Neben diesen eher klassischen regelungstechnischen Ansätzen welche unter anderem mit PID-Reglern oder Modellprädiktiver Regelung \ref{link:MPC} arbeiten gibt es Ansätze das Problem der Quadcopterregelung mit Methoden des tiefen verstärkenden Lernens zu lösen \ref{link:ReinforcementLearningQuadcopter}. Die hohe Dynamik des Felds lässt hoffen, dass die Neuentwicklung eines Quadcopters von Grund auf besser realisierbar ist als noch vor ein paar Jahren da der Zugang zu relevanten Informationen und Programmbeispielen verbessert worden sein könnte. Um dieser Frage nachzugehen wurden in dieser Arbeit praktisch versucht ein Quadcopter von null aufzubauen.\\
Primärziel ist einen Quadcopterprototyp mit allen relevanten Systemkomponenten zu produzieren. Sekundärziel ist die Regelung des Quadcopter eigenständig zu entwickeln wobei verschiedene Methoden analysiert und erprobt werden sollen.