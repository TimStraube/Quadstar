\section{Zusammenfasssung}
Entwickelt wurde ein System welches sich aus Impuls-, Energie- und Informationsflussteilsystemen zusammensetzt und eignet zur Simulation, Optimierung und Entwicklung von Quadcopterreglern. Das Gesamtsystem kann auch von Nutzern mit geringer Programmiererfahrung bedient werden.\\
Die Entwicklung von serienreifen Reglern auf Basis von der Software ist noch in ihren Anfängen. So wurden diverse Modelle antrainiert welche teilweise erste Anzeichen von Qualität zeigen. Weiteres Hyperparametertuning ist notwendig um die gelernten Regler auf ein Markttaugliches Niveau zu bringen. Auch sind die Regler für sich genommen unspektakulär. Die Integration von weitere Funktionen wie Trajektorien-Folgeregelung oder Multi-Agenten-Simulationen können zukünftige Schritte sein, um den Nutzen des Gesamtsystems weiter zu steigern.\\
Die Entwicklung des physikalisch Quadcopters ist in den letzten Zügen, noch zu optimieren beziehungsweise zu debuggen sind die Drehzahlregler sowie die Parametrisierung der Sensordatenfusion. Die Integration von einem Regler welcher mit gelernten Parameter oder direkt mit einem Ende-zu-Ende-Modell arbeiten ist auch schon in Aussicht.